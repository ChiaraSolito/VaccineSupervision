\documentclass{article}

\usepackage[T1]{fontenc}
\usepackage[utf8]{inputenc}
\usepackage{graphicx}
\usepackage{caption}
\usepackage[font=small,labelfont=bf]{caption}
\usepackage{booktabs, siunitx}
\usepackage{tikz}
\usepackage{tikz-qtree}
\usepackage{pifont}
\usepackage[margin=0.90in]{geometry}
\usepackage{etoolbox,titling}
\usepackage{enumitem}
\usepackage{fancyhdr}
\usepackage{soulutf8}
\usepackage{epigraph}
\usepackage{amssymb}
\usepackage{amsmath}

\pagestyle{fancy}
\fancyhf{}
\rhead{Virginia Filippi e Chiara Solito}
\lhead{VaccineSupervision}
\rfoot{Pagina \thepage}
\lfoot{Bioinformatica - A.A. 2021/22}
\usetikzlibrary{trees}
\tikzstyle{every node}=[draw=black,thick,anchor=west]
\newcommand{\angstrom}{\mbox{\normalfont\AA}}

\begin{document}
\newcommand\tab[1][0.3cm]{\hspace*{#1}}


\begin{titlepage}
    \begin{center}
        \vspace*{1cm}
            
        \Huge
        \textbf{Vaccine Supervision}
            
        \vspace{0.5cm}
        \LARGE
        Test dell'Utente Dottore
            
        \vspace{1.5cm}
            
        \textbf{A cura di Virginia Filippi e Chiara Solito}

        \vspace{0.8cm}

            
        \Large
        Corso di Laurea in Bioinformatica\\
        Università degli studi di Verona\\
        A.A. 2021/22
            
    \end{center}
\end{titlepage}

\section{Traccia su cui basare l'idea del funzionamento}
    Di seguito riportiamo la traccia dell'elaborato per quel che riguarda le funzionalità dell'utente di tipo Dottore.
    \begin{quotation}
       \textit{Si vuole progettare un sistema software per gestire le segnalazioni di reazioni avverse (ad esempio, asma, dermatiti, insufficienza renale, miocardiopatia, \dots) da vaccini anti-Covid.\\
        Ogni segnalazione è caratterizzata da un codice univoco, dall'indicazione del paziente a cui fa riferimento, dall'indicazione della reazione avversa, dalla data della reazione avversa, dalla data di segnalazione, e dalle vaccinazioni ricevute nei due mesi precedenti il momento della reazione avversa.
        Per ogni paziente sono memorizzati: un codice univoco, l'anno di nascita, la provincia di residenza e la professione.\\
        Per ogni paziente è possibile memorizzare gli eventuali fattori di rischio presenti (paziente fumatore, iperteso, sovrappeso, paziente fragile per precedenti patologie cardiovascolari/oncologiche), anche più d'uno. Ogni fattore di rischio è caratterizzato da un nome univoco, una descrizione e il livello di rischio associato. Per ogni paziente è, inoltre, memorizzata l'intera storia delle sue vaccinazioni precedenti, anti-Covid-19 e antinfluenzali.\\
        Ogni vaccinazione è caratterizzata da: paziente a cui si riferisce, segnalazioni a cui è legata, vaccino somministrato (AstraZeneca, Pfizer, Moderna, Sputnik, Sinovac, antinfluenzale, \dots), tipo della somministrazione (I, II, III o IV dose, dose unica), sede presso la quale è avvenuta la vaccinazione e data di vaccinazione. Per ogni reazione avversa sono memorizzati un nome univoco, un livello di gravità (da 1 a 5) e una descrizione generale, espressa in linguaggio naturale. Una reazione avversa può essere legata a molte segnalazioni. Per ogni paziente sono memorizzati il numero di reazioni avverse segnalate ed il numero di vaccinazioni ricevute.\\
        Il sistema deve supportare i medici che effettuano la segnalazione. \textbf{Dopo opportuna autenticazione, il medico viene introdotto ad una interfaccia che permette l'inserimento dei dati delle reazioni avverse e dei pazienti. Il codice univoco dei pazienti è gestito dal sistema, che tiene traccia dei pazienti indicati da ogni medico. Ogni medico vede solo i codici identificativi dei pazienti, dei quali ha già segnalato qualche reazione avversa, e le relative informazioni.}}
    \end{quotation}

\section{Passi che si richiede di effettuare}

    \subsection*{Tester 1}
    Dopo aver avviato l'applicazione si chiede all'utente di effettuare i seguenti passaggi:
    \begin{enumerate}
        \item Effettuare un login tramite le credenziali: username=\textit{doc12}, password=\textit{PROVA}.
        \item Verificare che l'accesso è negato.
        \item Effettuare un login tramite le credenziali: username=\textit{doc3}, password=\textit{DOC3}.
        \item Verificare che l'accesso è eseguito come dottore.
        \item Accedere alla lista pazienti tramite l'apposito bottone.
        \item Verificare che questa sia vuota.
        \item Tornare alla pagina principale.
        \item Accedere al form di inserimento dei report
        \item Nella prima tab cliccare il bottone di inserimento di un nuovo paziente.
        \item Inserire le seguenti informazioni\\
                Anno di nascita: \textit{1959}, Provincia: \textit{Lecco}, Professione:\textit{Impiegato}, scegliere tra i fattori di rischio il fattore esistente \textit{Colesterolo alto}.
        \item Passare alla tab successiva e inserire come data reazione, dall'apposito calendario il \textit{22 Maggio del 2022}.
        \item Cliccare il bottone di inserimento di una nuova reazione.
        \item Inserire le seguenti informazioni\\ 
                Nome: \textit{Trombosi}, Descrizione: \textit{Presenza di un coagulo di sangue, che aderisce alle pareti non lesionate dei vasi.}, Gravità: \textit{4}.
        \item Passare alla tab successiva e scegliere dai vaccini: \textit{Antinfluenzale A/Victoria/2570/2019}
        \item Inserire nei campi di testo:\\
                Sito della vaccinazione: \textit{Varenna}, Data della vaccinazione: \textit{4/12/2022}
        \item Cliccare su inserisci.
        \item Inserire nuovamente un vaccino: questa volta si scelga il vaccino \textit{Pfizer}
        \item Apparirà un nuovo bottone per scegliere il tipo di somministrazione, si clicchi su: \textit{Prima dose}
        \item Inserire nei campi di testo:\\
            Sito della vaccinazione: \textit{Varenna}, Data della vaccinazione: \textit{2/16/2022}
        \item Cliccare su inserisci.
        \item Inserire nuovamente un vaccino: questa volta si scelga il vaccino \textit{Moderna}
        \item Apparirà un nuovo bottone per scegliere il tipo di somministrazione, si clicchi su: \textit{Seconda dose}
        \item Inserire nei campi di testo:\\
            Sito della vaccinazione: \textit{Lecco}, Data della vaccinazione: \textit{4/31/2022}
        \item Cliccare su inserisci.
        \item Cliccare su conferma.
        \item Apparirà ora un avviso di conferma, cliccare su OK.
        \item Si tornerà sutomaticamente nella pagina iniziale. 
        \item Accedere alla lista pazienti tramite l'apposito bottone.
        \item Verificare la presenza di un nuovo paziente
        \item Cliccare sul bottone "Info".
        \item Dalla pagina aperta scorrere tra le tab e verificare la correttezza delle informazioni inserite.
        \item Tornare indietro e poi ancora fino alla pagina principale.
        \item Accedere di nuovo al form di inserimento dei report.
        \item Selezionare il paziente già esistente nella prima tab.
        \item Inserire come data della reazione: \textit{15 Giugno 2022}
        \item Selezionare dalle reazioni esistenti la reazione \textit{Eruzione cutanea}
        \item Nella tab successiva, inserire una nuova vaccinazione:\\
            si scelga il vaccino \textit{Moderna}
        \item Apparirà un nuovo bottone per scegliere il tipo di somministrazione, si clicchi su: \textit{Dose booster}
        \item Inserire nei campi di testo:\\
            Sito della vaccinazione: \textit{Varenna}, Data della vaccinazione: \textit{6/3/2022}
        \item Cliccare su conferma.
        \item Verificare nuovamente le informazioni inserite dalla pagina di visualizzazione dei pazienti.
        \item Tornare alla pagina iniziale.
        \item Effettuare il logout.
    \end{enumerate}

    \subsection*{Tester 2}
    Dopo aver avviato l'applicazione si chiede all'utente di effettuare i seguenti passaggi:
    \begin{enumerate}
        \item Effettuare un login tramite le credenziali: username=\textit{doc2}, password=\textit{PROVA}.
        \item Verificare che l'accesso è negato.
        \item Effettuare un login tramite le credenziali: username=\textit{doc1}, password=\textit{DOC1}.
        \item Verificare che l'accesso è eseguito come dottore.
        \item Accedere alla lista pazienti tramite l'apposito bottone.
        \item Verificare che questa sia popolata.
        \item Tornare alla pagina principale.
        \item Accedere al form di inserimento dei report
        \item Nella prima tab cliccare il bottone di inserimento di un nuovo paziente.
        \item Inserire le seguenti informazioni\\
                Anno di nascita: \textit{1980}, Provincia: \textit{Salerno}, Professione:\textit{Ristoratore}, 
                inserire un nuovo fattore di rischio: Nome: \textit{Sovrappeso}, Descrizione: \textit{Paziente con un peso superiore a quello indicato per eta e statura}, Livello di rischio: \textit{3}. Cliccare il bottone di inserimento.
        \item Selezionare ora il fattore di rischio aggiunto dal menù a tendina.
        \item Passare alla tab successiva e inserire come data reazione, dall'apposito calendario il \textit{30 Maggio del 2022}.
        \item Selezionare dalle reazioni esistenti la reazione \textit{Vomito}
        \item Passare alla tab successiva e scegliere dai vaccini: \textit{Antinfluenzale A/Darwin/6/2021}
        \item Inserire nei campi di testo:\\
                Sito della vaccinazione: \textit{Salerno}, Data della vaccinazione: \textit{5/20/2022}
        \item Cliccare su inserisci.
        \item Inserire nuovamente un vaccino: questa volta si scelga il vaccino \textit{Jannsen}
        \item Apparirà un nuovo bottone per scegliere il tipo di somministrazione, si clicchi su: \textit{Unica}
        \item Inserire nei campi di testo:\\
            Sito della vaccinazione: \textit{Salerno}, Data della vaccinazione: \textit{4/16/2022}
        \item Cliccare su inserisci.
        \item Cliccare su conferma.
        \item Apparirà ora un avviso di conferma, cliccare su OK.
        \item Si tornerà sutomaticamente nella pagina iniziale. 
        \item Accedere alla lista pazienti tramite l'apposito bottone.
        \item Verificare la presenza di un nuovo paziente
        \item Cliccare sul bottone "Info".
        \item Dalla pagina aperta scorrere tra le tab e verificare la correttezza delle informazioni inserite.
        \item Tornare indietro e poi ancora fino alla pagina principale.
        \item Accedere di nuovo al form di inserimento dei report.
        \item Selezionare il paziente con codice 20 nella prima tab.
        \item Inserire come data della reazione: \textit{23 Giugno 2022}
        \item Cliccare il bottone di inserimento di una nuova reazione.
        \item Inserire le seguenti informazioni\\ 
                Nome: \textit{Dermatite}, Descrizione: \textit{Infiammazione della pelle causata dalla reazione a determinate sostanze chimiche o naturali (dette allergeni).}, Gravità: \textit{2}.
        \item Nella tab successiva, inserire una nuova vaccinazione:\\
            si scelga il vaccino \textit{Moderna}
        \item Apparirà un nuovo bottone per scegliere il tipo di somministrazione, si clicchi su: \textit{Dose booster}
        \item Inserire nei campi di testo:\\
            Sito della vaccinazione: \textit{Torino}, Data della vaccinazione: \textit{6/20/2022}
        \item Cliccare su conferma.
        \item Verificare nuovamente le informazioni inserite dalla pagina di visualizzazione dei pazienti.
        \item Tornare alla pagina iniziale.
        \item Effettuare il logout.
    \end{enumerate}

\section{Tabella di valutazione}
    Si chiede ad entrambi i partecipanti di compilare la tabella di valutazione data assieme a questo modulo.

\end{document}